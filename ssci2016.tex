\documentclass[conference]{IEEEtran}
\IEEEoverridecommandlockouts

\usepackage[british]{babel}
\usepackage[noadjust]{cite}
\usepackage{graphicx}
\usepackage[hyphens]{url}
\usepackage{paralist}
\usepackage{booktabs}
%\usepackage{float}
\usepackage[pdftex,colorlinks=true]{hyperref}

% reduce indent on all paralist environments
%\setdefaultleftmargin{0.5cm}{}{}{}{}{}

\begin{document}

% paper title
\title{Engagement and Interactions of Language Communities on Twitter}


% author names and affiliations
% use a multiple column layout for up to three different
% affiliations
\author{\IEEEauthorblockN{Nabeel Albishry}
\IEEEauthorblockA{Department of Computer Science\\
University of Bristol\\
Bristol, UK\\
Email: n.albishry@bristol.ac.uk}
\and
\IEEEauthorblockN{Theo Tryfonas}
\IEEEauthorblockA{Department of Civil Engineering\\
University of Bristol\\
Bristol, UK\\
Email: theo.tryfonas@bristol.ac.uk}
\and
\IEEEauthorblockN{Tom Crick}
\IEEEauthorblockA{Department of Computing\\
Cardiff Metropolitan University\\
Cardiff, UK\\
Email: tcrick@cardiffmet.ac.uk}}

% conference papers do not typically use \thanks and this command
% is locked out in conference mode. If really needed, such as for
% the acknowledgment of grants, issue a \IEEEoverridecommandlockouts
% after \documentclass

% use for special paper notices
%\IEEEspecialpapernotice{(Invited Paper)}

% make the title area
\maketitle


\begin{abstract}
While emerging research is providing insight into the factors that
promote the propagation of information in online social networks
following significant events, such as rioting and terrorism, this
paper evaluates the extent to which different language communities
engage and interact. We present our analysis of online interactions in
various languages that took place on the social networking site
Twitter during the Baltimore protests in April 2015 in the USA.

By utilising language information from user profiles
({\emph{N}}=716,494) and status updates ({\emph{N}}=1,257,065)
relating to the Baltimore protests to identify and categorise
communities, we are able to provide insight into the pattern of their
interactions, as well as constructing their network graphs. The
results show that the nature of the event is reflected on the
engagement degree and wider interaction of communities. This analysis
of language communities may also help in deciding which group of users
to engage with, and hence increase the chance of influential action
when participating on Twitter conversations.
\end{abstract}

% For peer review papers, you can put extra information on the cover
% page as needed:
% \ifCLASSOPTIONpeerreview
% \begin{center} \bfseries Keywords \end{center}
% \fi
%
% For peerreview papers, this IEEEtran command inserts a page break and
% creates the second title. It will be ignored for other modes.
%\IEEEpeerreviewmaketitle

% tweak these at the end
% \begin{IEEEkeywords}
% Social network analysis, Twitter, language communities, information
% flows, information propagation, social media, networks.
% \end{IEEEkeywords}


\section{Introduction}\label{intro}

\subsection{Online Social Networks}
 
In recent years, online social networks (OSNs) have been utilised as
means to express ideas and opinions, spread information about events,
or even stimulate and propagate calls for civic engagement and
societal action. Social networking sites such as Twitter, Facebook,
LinkedIn and YouTube have also empowered individuals to promote their
viewpoints and interests -- professional or otherwise -- to a broad
and diverse global audience. The engagement of certain demographics
with social networks offers the opportunity for researchers interested
in observing and interpreting society to apply established theory and
methods to an emerging digital culture.

To satisfy the demand for various types of communities, interactions
and engagement, there are now vast numbers of social media sites and
platforms\footnote{This list is by no means exhaustive:
\url{http://en.wikipedia.org/wiki/List of social networking
websites}}, along with a number of attempted categorisations. By 2018,
there will be an estimated 2.5 billion active social network users (up
from 1.9 billion in 2014); they are producing massive amounts of data
(volume) on a real-time basis (velocity) with implicit sociological
attributes such as beliefs, opinions, sentiments, behaviours,
structures and influences (variety)~\cite{burnap-et-al:2015}. These
data exhibit the key traits of what is now referred to as big data:
volume, velocity and variety~\cite{postsm:2014}. In this age of big
data and an increasingly interconnected digital society, there is a
new challenge -- the application of robust and scalable methods and
tools that can be applied to digitised social behaviour generated via
social networks so as to be able to efficiently analyse big social
data to provide insight into real-world events and
actions~\cite{lazer-et-al:2009,burnap-et-al:2015}.

% Behaviour and interaction have been empirically studied for centuries
% by social scientists, with rigorously applied methods and analyses
% being developed to yield meaningful results.

Recent
work~\cite{blamey-et-al-2012,schwartz-et-al:2013,blamey-et-al-2013,oatley+crick:2014,oatley-et-al-soccogcomp2015}
has analysed what people say on social media to identify distinctive
words, phrases, and topics as functions of known attributes of people
such as gender, age, location, or psychological characteristics. This
can thus be collated and aggregated, inferring gender, age, location
and sentiments, from social media data. Potential negative
implications of these approaches include the fact that they can be
easily applied to large numbers of people or groups in society without
obtaining their explicit consent or even being aware it is being
done. Data-driven commercial companies, governmental entities, or
even one's followers or friends are able to use software to infer
personality and other attributes -- such as sexual orientation or
political affiliations -- that an individual may have decided not to
share~\cite{lambiotte+kosinski:2014,postsm:2014}.

There are various projects that have used Twitter corpora and related
datasets to make predictions about
elections~\cite{tumasjan-et-al:2010}, stock
markets~\cite{zhang-et-al:2011}, and crimes and
policing~\cite{gerber:2014,oatley+crick_fosintsi2014,oatley+crick:2015}. Twitter
played an important role during what was then known as the ``Arab
Spring'', which has been extensively examined in the social network
analysis
domain~\cite{lotan-et-al:2011,howard-et-al:2011,comunello+anzera:2012,wolfsfeld-et-al:2013,bruns-et-al:2013}.
While the use of Twitter data has been demonstrated to provide insight
-- and sociologically relevant demographics~\cite{sloan-et-al:2013} --
into major social and physical events such as
riots~\cite{procter-et-al:2013} and terror
attacks~\cite{burnap-et-al:2014}, often all is not what it may seem;
for instance many tweets may not a crowd make~\cite{liang-et-al:2013}.

\subsection{Languages and Communities}

Despite the widespread engagement with Twitter globally, little
research has investigated the differences amongst users of various
languages; there is a tendency to assume that the behaviours of
English users generalise to other language
users~\cite{hong-et-al:2011}. Language has featured as a facet of
research on the geographies of Twitter
networks~\cite{takhteyev-et-al:2012}, especially whether offline
geography still matter in online social
networks~\cite{kulshrestha-et-al:2012}. Linguistic-inspired studies
have been done on hashtags~\cite{cunha-et-al:2011}, as well as the
volume and proportional of tweets in English and Arabic, as part of an
analysis of the Arab Spring~\cite{bruns-et-al:2013}. Nevertheless,
language is clearly a vital component of affiliation and discourse on
the web~\cite{zappavigna+martin:2012}, with the creation and curation
of emerging multi-lingual networks and communities, representing
well-established creative, cultural and socio-economic norms,
including for minority languages such as Welsh~\cite{gj+uj:2013}.

\subsection{Social Network Analysis}

In the social network analysis (SNA) domain, centrality measures
provide the ability to assess network graphs that are constructed from
collected data (for example, tweets). Selection of these centrality
measures is dependent on the goal of the analysis; for example, the
degree of node helps to identify nodes with high number of connections
within the
network~\cite{borgatti+everett:2000,rombach-et-al:2014,liu-et-al:2014}.
In a representation of a real world network, this metric may help to
identify highly connected persons, such as political leaders, sports
stars or celebrities, who are potential ``information
spreaders''~\cite{cha-et-al:2012,borge-holthoefer-et-al:2012,zhang-et-al:2016}.
Centrality measures such as degrees, betweenness, clustering
coefficient, modularity and cliques have been used in many projects to
measure influence or detect the emergence of new
communities~\cite{willis-et-al:2015,oatley+crick:2015}.

Clustering users in communities has been an important analytic factor
in social networking analysis; numerous work has focused on clustering
users based on their locations. However, for the sake of anonymity,
many users tend not to disclose information about their identity, such
as locations~\cite{kang-et-al:2013}. It has also been reported in the
literature that geotagged tweets are generally low in
number~\cite{morstatter-et-al:2013,tan-et-al:2013,kumar-et-al:2014},
the exponential growth in social media over the past decade has been
joined by the rise of location as a central organising
theme~\cite{liang-et-al:2013} of how users engage with online
information services and, more importantly, with each
other~\cite{cheng-et-al:2010,caverlee-et-al:2013}.

\subsection{Users and Location}

It is important to understand how geotagging works in Twitter. The
`place' entity included in a Twitter status does not necessarily
indicate precisely where the actual posting was made, as stated in the
Twitter API
documentation\footnote{\url{https://dev.twitter.com/overview/api/places}}:

\begin{quotation} ``{\emph{Tweets associated with places are not
necessarily issued from that location but could also potentially be
about that location.}}
\end{quotation}

For the sake of anonymity many users tend not to disclose information
about their identity, particularly locations; this has also been
supported by the literature that geotagged tweets are generally low in
number~\cite{kang-et-al:2013}. An alternative location-based option to
consider is based on profile location, but this may not serve the need
for location clustering for a multitude of reasons, especially with a
significant proportion of Twitter users not setting their profile
location~\cite{graham-et-al:2014} (discussed in more detail in
Section~\ref{locations}).

\subsection{Overview of Paper}

The techniques we introduce in this paper are based on language settings
in users' profiles and those for statuses\footnote{The term `status'
is a generic term used to refer to any Twitter post (tweet, retweet,
reply, or quote).}. The remainder of this paper is organised as
follows: Section~\ref{context} introduces the context of the Baltimore
protest case study and the development of events in alignment with
activity on Twitter. In Section~\ref{langcomm} we present the
techniques we have used to identify and analyse language communities
and networks, along with the results. Finally,
Section~\ref{conclusions} concludes the paper with a wider discussions
and a summary of the potential application of our approach.


\section{Context and Events Timeline}\label{context}

Following the peaceful funeral of Freddie Gray that took place on the
morning of Monday 27 April 2015 in Baltimore, Maryland, USA, a protest
hit the city. According to the timeline published on the CNN website
``{\emph{The city exploded on Monday after the funeral of Freddie
Gray, a 25-year-old black man who mysteriously died on April 19, a
week after Baltimore Police arrested
him.}}''~\cite{baltimorewiki:2015}. The nature of the Baltimore protests
is a good representation of a planned event in which a sudden
escalation of violence hits a geographical area. The event manifested
itself on Twitter as {\texttt{\#BaltimoreRiots}}, and resulted in more
than 1,250,000 status updates.

Figure~\ref{fig:overallactivity} and Figure~\ref{fig:mainevents}
present how the event manifested itself on Twitter once a ``purge'' was
scheduled. We can see that what was happening on the ground was
quickly reflected on the activity in Twitter, as
Figure~\ref{fig:mainevents} indicates. More detailed analysis reveals
that within one hour the topic started to go ``viral''; more
precisely, at approximately 15:00 at which the ``purge'' was
scheduled. The topic jumped from roughly 1,200 to 8,000 tweets per
hour. Then, it peaked with 98,000 between 22:00 and 23:00.

\begin{figure}[!htb]
\centering
\includegraphics[width=\columnwidth]{images/overallactivity.png}
\caption{Overall activity for {\texttt{\#BaltimoreRiots}}.}
\label{fig:overallactivity}
\end{figure}

\begin{figure}[!htb]
\centering
\includegraphics[width=\columnwidth]{images/mainevents.png}
\caption{Main events during the lifetime of {\texttt{\#BaltimoreRiots}}.}
\label{fig:mainevents}
\end{figure}


\section{Language Communities}\label{langcomm}

Analysis of language communities begins with two basic techniques. The
first is to classify statuses based on their languages. The status
language is extracted from the `{\emph{lang}}' entity inside status
objects. Language used in posting defines which community the status
was meant for; a tweet written in Turkish, for example, is meant for
the Turkish-speaking community. Output from this will be referred to
as `posting communities'. The second analysis is to classify users
into different communities based on their profile languages,
regardless of the posting language they used. Output from this
technique will be referred to as `profile communities'. As we will see
in the following sections, a posting community does not necessarily
indicate the profile community for a user. Therefore, the second step
is to examine the relationship between profile and posting
communities. We will also explore relationships amongst profile
communities, in term of action-reaction. We will investigate the
intra- and inter-profile communities interactions by constructing
network graphs and generate a visual representation.

\subsection{Locations}\label{locations}

As mentioned in Section~\ref{intro}, for the sake of anonymity many
users tend not to disclose information about their identity,
particularly locations; this has also been supported by the literature
that geotagged tweets are generally low in
number~\cite{kang-et-al:2013}. We took the step to verify this claim
in our datasets; in the best cases, the ratio of geotagged tweets did
not exceed 2\%. In the case of the {\texttt{\#BaltimoreRiots}}
dataset, only ~1\% of collected statuses were associated with
places. Moreover, out of this geotagged subset, only 4\% were
associated with the city where the event took place (Baltimore).

An alternative location-based option to consider is based on profile
location, but it still does not serve the need for location clustering
for a multitude of reasons. Firstly, we found that less than 45\% of
users have set their profile location, which is in line with other
studies~\cite{graham-et-al:2014}. Secondly, although Twitter suggests
certain presets for setting profile location, users are given the
option to enter any text they wish; this results in a considerable
amount of noise.

\subsection{Posting Communities}

In the {\texttt{\#BaltimoreRiots}} case, there were 38 posting
languages. As we can see in Figure~\ref{fig:langfreq}, English was the
dominating language by far. Interestingly, results also show that
language of more than 41,000 (~3\%) statuses could not be
identified. When investigated, those statuses mostly do not contain
text other than hashtags, pictures or URLs. Although, this is not a
big portion, it came second after English. Although this category
shows an interesting case in which qualitative content analysis would
be involved, it is beyond this study and will not be covered here.

\begin{figure}[!htb]
\centering
\includegraphics[width=\columnwidth]{images/langfreq.png}
\caption{Most frequently used languages in
  {\texttt{\#BaltimoreRiots}}: 
({\emph{en:}} English; {\emph{es:}} Spanish; {\emph{tr:}} Turkish;
  {\emph{fr:}} French; {\emph{en-gb:}} British English; {\emph{ar:}}
  Arabic; {\emph{de:}} German; {\emph{ru:}} Russian; {\emph{it:}}
  Italian; {\emph{pt:}} Portuguese)}
\label{fig:langfreq}
\end{figure}


\subsection{Users' Language Communities}

In the majority of cases, users choose to pick a language for their
Twitter profile settings. In our dataset we found that out of 716,494
users, only 45 had not chosen any language. However, the language
entity returned by the API for those cases is the initial placeholder
text ``{\emph{Select Language...}}'' or a translated version that might provide
hints regarding the user language
community. Figure~\ref{fig:activecomm} shows that about 94\% of the
users came from `{\emph{en}}' profile community.

\begin{figure}[!htb]
\centering
\includegraphics[width=\columnwidth]{images/activecomm.png}
\caption{Top 10 most active communities in {\texttt{\#BaltimoreRiots}}}
\label{fig:activecomm}
\end{figure}

From these two outputs, we can see that nearly all of the topic
activity came from one particular community using one particular
language. This extreme pattern may accompany extreme and
geographically constrained real world events such as riots and terror
attacks.


\subsection{Profile-Posting Graph}

To investigate whether the `{\emph{en}}' posting community is linked to
particular profile communities, we constructed a bipartite graph as
presented in Figure~\ref{fig:profilepostinggraph}, representing the
profile-posting language network. In this graph, nodes that are
prefixed by ``{\emph{p\_}}'' represent profile language community, and
nodes that are prefixed by ``{\emph{s\_}}'' represent status language
community. Size of node represents the weighted degree, whereas colour
represents the outdegree; the darker the colour, the higher
outdegree. The graph confirms the domination pattern we highlighted
earlier; furthermore, it shows the relationships between the profile and
posting communities. 

\begin{figure*}[!h]
\centering
\includegraphics[width=\textwidth]{images/profilepostinggraph.png}
\caption{Profile-posting network graph}
\label{fig:profilepostinggraph}
\end{figure*}

\begin{figure*}[!h]
\centering
\includegraphics[width=\textwidth]{images/profileprofilegraph.png}
\caption{Profile-profile network graph}
\label{fig:profileprofilegraph}
\end{figure*}

For an extreme case of one dominating posting language, we wanted to
investigate participation of different communities. We thus filtered
out all non-`{\emph{en}}' statuses, and then identified different
profile communities with the resultant set. For each community, we
classified statuses into two sets: {\emph{actions}} and
{\emph{reactions}}; this result is shown in
Table~\ref{tbl:mostactive}. This shows the highest scoring
communities, where the first column represents the category of status
(action or reaction), community column represents profile language
community, and last one shows percentage of `{\emph{en}}' posts by
that community.

\begin{table}[!htb]
\centering
\begin{tabular}{@{}lcr@{}}
\toprule
\textbf{Category} & \textbf{Community} & \textbf{\%} \\ \midrule
Reaction & {\emph{en}} & 81.08 \\
Action & {\emph{en}} & 15.43 \\
Reaction & {\emph{es}} & 0.68 \\
Reaction & {\emph{fr}} & 0.59 \\
Reaction & {\emph{en-gb}} & 0.47 \\
Reaction & {\emph{tr}} & 0.19 \\
Reaction & {\emph{de}} & 0.18 \\
Reaction & {\emph{pt}} & 0.13 \\ \bottomrule
\end{tabular}
\caption{Activity and categories of most active profile language
  communities}
\label{tbl:mostactive}
\end{table}

From the results above we can infer that there is a dominating player
in both domains: posting languages and profile communities. Therefore,
for the case of {\texttt{\#BaltimoreRiots}}, we can conclude that the case was
substantially localised.


\subsection{Temporal Communities Activity}

We wished to explore communities' activity over time, apart from the
overall activity. Figure~\ref{fig:heatmap} represents a heat map for
posts per hour for each profile community over the lifetime of
{\texttt{\#BaltimoreRiots}}. This sort of mapping would help in
identifying times at which communities are active.

When refined with posting language, this technique would be useful in
identifying when to engage into conversations, which language
community to target, and by which language. For example, presuming
that we want to participate in a trending topic hashtag that we are
interested in; this analysis technique could help us make our post
more direct and focused. By finding which language is mostly used in
posting, we will be able to know in which language the tweet would be
more effective. Also, we might want to direct the message to certain
language community, be that to influence a very active or re-activate
a quiet one.

\begin{figure*}[!h]
\centering
\includegraphics[width=\textwidth]{images/heatmap.png}
\caption{Temporal activity of profile communities}
\label{fig:heatmap}
\end{figure*}


\subsection{Reaction Networks}\label{reactionnetworks}

Another important perspective to identify and capture is how different
profile communities relate to each other in terms of
reactions. Therefore, we constructed the graph in
Figure~\ref{fig:profileprofilegraph} to show interactions amongst
language communities. Edges are directed and are drawn from reacting
nodes (retweeter, or replier) to the original acting node
(tweeter). As we are not interested in reactions from community to
itself, we eliminated self-loop edges from the graph. Node size
represents weighted indegree; we have also generated the degree
measures for the various language communities, as shown in
Figure~\ref{fig:inoutdegree}.

\begin{figure*}[!htb]
\centering
\includegraphics[width=\textwidth]{images/indegreeoutdegree.png}
\caption{Degree measures for the various language communities}
\label{fig:inoutdegree}
\end{figure*}


\section{Conclusions}\label{conclusions}

This paper presented a study in identifying languages used, language
communities and their engagement and interactions on the Twitter
platform with respect to real world events, in this instance using the
Baltimore protests in the USA in April 2015. As we discussed in
Section~\ref{langcomm}, the nature of the event (e.g. being a local or
global) may be reflected on community conversations on Twitter. We
found that most of posting activity comes from the main community (the
language community in which the incident has happened or tightly
related to). This is especially the case when the online conversations
are triggered by a real world incident. The same for posting
languages, users mostly to use the language of the main
community. Although most of topic activity comes from that main
community, we noticed that, with local events, other communities work,
mostly, as information spreaders.

We also presented a network graph (using
Gephi\footnote{\url{https://gephi.org/}} and the Networkx Python
package\footnote{\url{https://networkx.github.io/}}) showing how
language communities relate to each other in the form of
action-reaction (action: tweet, reaction: retweet, reply,
quote). Another interesting graph that we produced to show relations
between profile and posting communities. We find that this graph is
important to facilitate comparing users defined profile language with
their posting language. Some event might be termed as `partially
scheduled' as their end was different to how they were planned in the
first place. In such tense situation, we noticed that diversity of
languages and communities are very low, and there always be a
dominating community and language.

The method we presented here can be used in identifying how
communities interact with one another, which ones are most active,
which languages are mostly used, and at what time. Applying these
techniques on data pouring from the Twitter Stream
API\footnote{\url{https://dev.twitter.com/streaming/overview}} would
be applicable to a wide number of domains. For example, these methods
can be used in social network marketing and publicity to increase the
probability of influential posts. In practice, for a given
{\texttt{\#<Brand>}}, by monitoring the activity of different language
community, one can decide the time to post well-tailored tweets
targeting certain communities. This can be fine-tuned further by
mentioning key players in that community, e.g. users with high
closeness scores.

Moreover, within certain contexts, the order of applying these two
classifications (posting and profile) will generate different results.
For example, taking one profile community and dividing it into
different posting communities shows the number of languages this
community may use, and hence degree of openness and reachability. A
possible scenario for governments, politicians or campaigners would be
to use this method to measure to what extent other languages are used
within a profile community. It may also show how users associate
themselves with one community in their profile while using other
languages. Monitoring unusual activity for secondary languages may
help to uncover important messages or opinions that could not be
openly expressed, for a variety of reasons, to the rest of the profile
community.

For the social network analysis domain, this method provides a
different perspective for influence analysis. Endorsement from
different profile communities cannot be measured similar to those
coming from the same community. For example, in a controversial Arabic
topic, we noticed that high support came from other profile
communities.

For future work, we plan to add further classifications to the
reaction network (as presented in Section~\ref{reactionnetworks}). We
believe that differentiation between endorsements (e.g. retweets) and other
reactions may provide further insight into the networks and
communities. Furthermore, we will apply the methods presented in this
paper on other high-profile event/discussion datasets in different
domains or contexts, such as for sports, music contests and
humanitarian actions.


\section*{Acknowledgment}
\addcontentsline{toc}{section}{Acknowledgment}

This work has been supported by a doctoral research scholarship for
Nabeel Albishry from King Abdulaziz University, Kingdom of Saudi
Arabia.


%\newpage

% trigger a \newpage just before the given reference
% number - used to balance the columns on the last page
% adjust value as needed - may need to be readjusted if
% the document is modified later
\IEEEtriggeratref{29}
% The "triggered" command can be changed if desired:
%\IEEEtriggercmd{\enlargethispage{-5in}}

% references section
\bibliographystyle{IEEEtran}
\bibliography{ssci2016}

% that's all folks
\end{document}


