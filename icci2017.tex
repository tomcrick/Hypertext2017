\documentclass{llncs}

\usepackage{graphicx}
\usepackage{url}

% natbib for refs
\usepackage[numbers,sort]{natbib} 

\begin{document}

\title{``{\emph{Come Together!}}'': Interactions of Language Networks and Multilingual Communities on Twitter}

\author{Nabeel Albishry\inst{1}\thanks{This work has been supported by a doctoral research scholarship for
Nabeel Albishry from King Abdulaziz University, Kingdom of Saudi
Arabia.} \and Theo Tryfonas\inst{1} \and Tom
  Crick\inst{2}}

% \thanks{{\emph{N.B.}} The first part of the title of this paper was
% taken from the motto of the 2016 Eurovision Song Contest, which along
% with the theme artwork was said to be ``inspired by the dandelion,
% symbolising the power of resistance and resilience but also of
% regeneration''.}

\institute{Department of Computer Science, University of Bristol, UK\\\email{\{n.albishry,theo.tryfonas\}@bristol.ac.uk}
\and 
Department of Computing, Cardiff Metropolitan University, Cardiff, UK\\\email{tcrick@cardiffmet.ac.uk}}

\maketitle

\begin{abstract}
Emerging tools and methodologies are providing insight into the
factors that promote the propagation of information in online social
networks following significant activities, such as high-profile
international social or societal events; this paper provides insight
into how people are linked, by how different language communities
engage and interact. We present our analysis of two significant online
interactions in various languages that took place on the social
networking site Twitter: during the Baltimore protests in April 2015
in the USA and the Eurovision Song Contest in May 2016.

By utilising language information from user profiles (Baltimore:
{\emph{N}}=716,494; Eurovision: {\emph{N}}=1,226,959) and status
updates (Baltimore: {\emph{N}}=1,257,065; Eurovision:
{\emph{N}}=7,926,746) to identify and categorise communities, we are
able to provide insight into the pattern of their interactions, as
well as constructing their network graphs to shed light on these
multilingual community. The results show that the nature of the event
is reflected on the engagement degree and wider interaction of
communities, as well as indicating the participation pattern of
multilingual users. This analysis of language communities may also
help in deciding which group of users to engage with -- and hence
increase the chance of influential actions -- when participating in
large-scale Twitter conversations.
\end{abstract}

\section{Introduction}

\section{Conclusions}

% bib
\bibliographystyle{abbrvnat}
\bibliography{icci2017}

\end{document}
